\documentclass[10pt]{beamer}

\usetheme[%
    progressbar=frametitle,
    block=fill,
    numbering=fraction,
    footer=crumbs,
    sectionpage=numbered,
    subsectionpage=none,
    titleformattitle=smallcaps,
    titleformatsubtitle=smallcaps,
    %%% new options for UR2 theme:
    maincolor=red,
]{metropolis-ur2}

\usepackage{hyperref}
\hypersetup{
    colorlinks=true,
    linkcolor=.,
    filecolor=ur2Turquoise,
    urlcolor=ur2Bleu,
    pdftitle={M1 MAS - Python - Programmation Orientée Objet},
    pdfpagemode=FullScreen,
}

% \usepackage{biblatex}
% \addbibresource{example.bib}

\usepackage{appendixnumberbeamer}
\usepackage{adjustbox}
\usepackage{booktabs}
\usepackage{fontawesome5}
\usepackage{graphicx}
\usepackage{array}
\usepackage{tabularx}
\usepackage{myminted}
% \usepackage[cache=false]{minted}

\title[M1 MAS -- Python -- POO]{CM 8 - La Programmation Orientée Objet}
\subtitle{Programmation Python -- Master 1 MAS}
\author{Romain Tavenard}
\date{2022}
\institute{%
\hypersetup{urlcolor=.}
\makebox[2.2ex][c]{\faEnvelope}\enspace\href{mailto:romain.tavenard@univ-rennes2.fr}{\texttt{romain.tavenard@univ-rennes2.fr}}\\%
% \makebox[2.2ex][c]{\faHome}\enspace\url{https://rtavenar.github.io/}%
}

\begin{document}

\maketitle

\section{Vocabulaire}

\begin{frame}[fragile]{Propriétes \& Méthodes}
  \begin{beamercodeblock}
    \begin{minted}[fontsize=\footnotesize]{python}
      class MaClasseTropClasse:
        def __init__(self):
          self.a = 123          # Propriété (== attribut)
        
        def f(self, b):         # Méthode
          return b + self.a - 4

        @property
        def double_de_a(self):  # Propriété (calculée)
          return 2 * self.a

      o = MaClasseTropClasse()
      print(o.a)
      print(o.f())
      print(o.double_de_a)
    \end{minted}
  \end{beamercodeblock}
\end{frame}

\begin{frame}[fragile]{Propriétes / Attributs de classe}
  \begin{beamercodeblock}
    \begin{minted}[fontsize=\footnotesize]{python}
class MaClasseTropClasse:
  n_objets_crees = 0

  def __init__(self):
    self.a = 123          # Propriété (== attribut)
    MaClasseTropClasse.n_objets_crees += 1

o = MaClasseTropClasse()
print(MaClasseTropClasse.n_objets_crees)

o2 = MaClasseTropClasse()
print(MaClasseTropClasse.n_objets_crees)
    \end{minted}
  \end{beamercodeblock}
\end{frame}

\begin{frame}[fragile]{Méthodes de classe}
  On peut aussi avoir des méthodes de classe : elles ne peuvent accéder qu'aux attributs de classe.

  \begin{beamercodeblock}
    \begin{minted}[fontsize=\footnotesize]{python}
class MaClasseTropClasse:
  n_objets_crees = 0
  def __init__(self):
    self.a = 123          # Propriété (== attribut)
    MaClasseTropClasse.n_objets_crees += 1
  @classmethod
  def print_n_objets(cls):
    print(cls.n_objets_crees)

o = MaClasseTropClasse()
o2 = MaClasseTropClasse()
MaClasseTropClasse.print_n_objets()
    \end{minted}
  \end{beamercodeblock}
\end{frame}

\begin{frame}[fragile]{Méthodes de classe, méthodes statiques}
  On peut aussi avoir des méthodes statiques : elles ne peuvent accéder ni aux attributs d'instance ni aux attributs de classe.

  \begin{beamercodeblock}
    \begin{minted}[fontsize=\footnotesize]{python}
class MaClasseTropClasse:
  n_objets_crees = 0
  def __init__(self):
    self.a = 123          # Propriété (== attribut)
    MaClasseTropClasse.n_objets_crees += 1
  @staticmethod
  def compare(objet1, objet2):
    print(f"Les valeurs respectives de a sont {objet1.a} et {objet2.a}")

o = MaClasseTropClasse()
o2 = MaClasseTropClasse()
o2.a = 55
MaClasseTropClasse.compare(o, o2)
    \end{minted}
  \end{beamercodeblock}
\end{frame}

\begin{frame}[fragile]{Exercice de synthèse}
  \begin{quote}
    Implémentez une classe \mintinline{python}|NombreComplexe| telle que l'on puisse exécuter le code suivant :
    \begin{beamercodeblock}
      \begin{minted}[fontsize=\footnotesize]{python}
        c0 = NombreComplexe(10, 0)  # 10
        c1 = NombreComplexe(0, 5)   # 5i
        c2 = NombreComplexe(3, -2)  # 3 - 2i
        print(c1.module())          # Affiche 5
        print(NombreComplexe.distance(c0, c1))     # Affiche 11.18...
      \end{minted}
    \end{beamercodeblock}
  \end{quote}
\end{frame}

\section{Héritage et conséquences}

\begin{frame}[fragile]{Héritage multiple}
  \begin{beamercodeblock}
    \begin{minted}[fontsize=\footnotesize]{python}
class ClasseA:
    def __init__(self):
        super().__init__()
        self.a = 0
class ClasseB:
    def __init__(self):
        super().__init__()
        self.b = 0
class ClasseC(ClasseA, ClasseB):
    def __init__(self):
        super().__init__()

obj_c = ClasseC()
print(obj_c.a, obj_c.b)
    \end{minted}
  \end{beamercodeblock}

  Attention, pour que cela fonctionne correctement, il faut que toutes les méthodes de la hiérarchie fassent bien appel au constructeur de leur(s) classe(s) parente(s).
\end{frame}

\begin{frame}[fragile]{Classe abstraite}
  Une classe abstraite :

  \begin{itemize}
    \item doit hériter de la classe \mintinline{python}|ABC| du module \mintinline{python}|abc| ;
    \item ne peut pas être instanciée ;
    \item peut être héritée ;
  \end{itemize}

  C'est un \alert{concept abstrait}.

  \pause

  De la même façon, il existe :

  \begin{itemize}
    \item des méthodes abstraites (décorateur \mintinline{python}|@abstractmethod|, défini dans le module \mintinline{python}|abc|) ;
    \item des propriétés calculées abstraites (décorateur \mintinline{python}|@abstractproperty|, défini dans le module \mintinline{python}|abc|).
  \end{itemize}
\end{frame}

\begin{frame}[fragile]{Exercice de synthèse}
  \begin{quote}
    Implémentez une classe abstraite \mintinline{python}|Polygone| et les classes \mintinline{python}|Rectangle| et \mintinline{python}|Carre| ayant :
    \begin{itemize}
      \item un attribut \mintinline{python}|cotes| qui est une liste des longueurs des côtés du polygone ;
      \item une méthode \mintinline{python}|perimetre| ;
      \item une méthode \mintinline{python}|aire|.
    \end{itemize}
  \end{quote}
\end{frame}

\end{document}
