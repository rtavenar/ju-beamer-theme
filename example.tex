\documentclass[10pt]{beamer}

\usetheme[%
    progressbar=frametitle,
    block=fill,
    numbering=fraction,
    footer=crumbs,
    sectionpage=numbered,
    subsectionpage=none,
    titleformattitle=smallcaps,
    titleformatsubtitle=smallcaps,
    %%% new options for UR2 theme:
    maincolor=red,
    titlelogo=urdeux,
    headlogo=urdeux,
    typeface=source,
]{metropolis-ur2}

\usepackage{hyperref}
\hypersetup{
    colorlinks=true,
    linkcolor=.,
    filecolor=ur2Turquoise,
    urlcolor=ur2Bleu,
    pdftitle={UR2 Beamer Theme},
    pdfpagemode=FullScreen,
}

\usepackage{biblatex}
\addbibresource{example.bib}

\usepackage{appendixnumberbeamer}
\usepackage{adjustbox}
\usepackage{booktabs}
\usepackage{fontawesome5}
\usepackage{graphicx}
\usepackage{array}
\usepackage{tabularx}
\usepackage{myminted}
% \usepackage[cache=false]{minted}

\title[Thème Beamer UR2]{Un thème Beamer Université de Rennes 2}
\subtitle{Metropolis customisé (dérivé du thème non officiel de Jönköping University)}
\author{Romain Tavenard}
\date{2022}
\institute{%
\hypersetup{urlcolor=.}
\makebox[2.2ex][c]{\faEnvelope}\enspace\href{mailto:romain.tavenard@univ-rennes2.fr}{\texttt{romain.tavenard@univ-rennes2.fr}}\\%
\makebox[2.2ex][c]{\faHome}\enspace\url{https://rtavenar.github.io/}%
}

\begin{document}

\maketitle

\begin{frame}[stretch=3]{A word of warning}

  \ifboolexpr{bool {xetex} or bool {luatex}}{
  }{
    \metroset{block=fill}
    \begin{alertblock}{Warning}
    You need to compile with XeLaTeX or LuaLaTeX to get proper font support! \textit{(LuaLaTeX is recommended on Overleaf.)}
    \end{alertblock}
  }

  \begin{itemize}
    \item This theme is \alert{not officially endorsed} by Jönköping University.
    \item It does \alert{not fully adhere} to JU's Graphic Manual.
    \item Use at your own discretion!
  \end{itemize}
\end{frame}

\begin{frame}{Contents}
\twocol{\tableofcontents[sections={1-2}]}{\tableofcontents[sections={3-5}]}

\end{frame}


\section{Features}
\subsection{How to use}

\begin{frame}[fragile,stretch=3]{How to use}
  \begin{itemize}
    \item Copy the contents of the \texttt{sty/} directory into your project.
  
    \item If you downloaded this template from GitHub, you also need to \href{https://intranet.hj.se/intranet/en/service-and-support/marketing-and-communication/graphic-profile/logotypes.html}{{\small\faExternalLink*}~download the JU~logotypes} from the intranet and extract the PNG~versions into the \texttt{sty/img/} directory.
  
    \item You can then use this theme by including:\medskip
  
    \begin{minted}{latex}
    \usetheme[maincolor=red]{metropolis-ur2}
    \end{minted}
    
    \medskip
    \item \alert{Use LuaLaTeX for best results.}
  \end{itemize}
\end{frame}

\subsection{Theme options}

{
\metroset{maincolor=grey}
\begin{frame}[fragile,stretch=3]{Options: Color theme}

    \begin{itemize}
    \item The \texttt{maincolor} option sets the background colors of title/standout frames as well as the      headline.\medskip
    
    It can be either \colorbox{ur2Rouge}{\textcolor{white}{\texttt{red}}} (default) or \colorbox{ur2Gris}{\textcolor{white}{\texttt{grey{\vphantom l}}}}.

    \item This slide shows what the theme looks like with \texttt{maincolor=grey}.
  
    \item You can change colors in the middle of your presentation via:\medskip

          \mint{latex}|   \metroset{maincolor=grey}|
    \end{itemize}
    
    \medskip
    \textit{Note: For changing the font theme, see the later \hyperlink{selectfonts}{\textcolor{ur2Bleu}{``Fonts'' section.}}}
\end{frame}
}

\begin{frame}[stretch]{Options: Logotypes}
  \begin{itemize}
    \item The \texttt{titlelogo} option sets the logotype to put on the title frame.  It can be either:
    \begin{itemize}
        \item \texttt{none}: no logo
        \item \texttt{urdeux}: UR2's main logotype \alert{\bf (this is the default)}
    \end{itemize}
    \item The \texttt{headlogo} option does the same thing, but for the smaller logotype in the headline.
    \item Logotypes from the other schools aren't currently included, but could of course also be added.
  \end{itemize}
\end{frame}

{\metroset{maincolor=grey}
\begin{frame}[fragile,stretch]{Aside --- Logotypes in Other Places}

The logotypes can be inserted with the following command:

\begin{center}
\begin{tabular}{lc}
    \mintinline{latex}{\urLogo} & \begin{minipage}{0.3\textwidth}\centering\urLogo[height=5ex]{}\end{minipage} \\[2ex]
\end{tabular}
\end{center}

Scaling options can be given as an argument, e.g.\ \mintinline{latex}{\urLogo[height=2cm]}.

\begin{itemize}
    \item To get the white versions of the logos, append \texttt{W} to the command; e.g. \mintinline{latex}{\urLogoW}.
\end{itemize}

\end{frame}
}

\begin{frame}{Options: Footer and Section Pages}
  In addition to all the options the standard \textsc{Metropolis} theme provides, I added the following options, implementing features from \textsc{colorful-dream}:

  \begin{itemize}
    \item \texttt{footer=crumbs} will put the ``breadcrumbs'' line in the footer with the current section/subsection titles.
    \item \texttt{sectionpage=numbered} will produce the section pages with numbered circles.
    \item \texttt{subsectionpage=numbered} will do the same thing for subsections \textit{(not enabled in this presentation)}.
  \end{itemize}
\end{frame}

\subsection{Layout}

\begin{frame}[stretch=5]{Layout: Typesetting the frame title}
  The frame title on the \alert{colored bar} at the top of the slide comes from the \textsc{Metropolis} theme.

  This can look nice, but \alert{does not match JU's own presentation templates.}
  
  However, you can also get something closer to JU's official style\ldots
\end{frame}

% \begin{frame}[fragile,stretch=1.5]
%   \juHeading{Typesetting the frame title}
  
%   A frame \emph{without} a title will look like this.
  
%   The heavy, all-uppercase heading can be produced via the command:
  
%     \mint{latex}|   \juHeading{A title in JU style}|

%   The automatic upper-casing might cause some issues with certain symbols or commands inside the heading.  If that's the case, you can use this version without the upper-casing magic:

%     \mint{latex}|   \juHeadingCased{A HEADING WITH ``QUOTES''}|
% \end{frame}

\begin{frame}[fragile,stretch=3.5]{Layout: Spacing}
It's nice to have less text per slide!

\begin{itemize}
    \item Frames now have an additional \alert{\texttt{stretch}} key with a stretch factor as an optional value (defaults to~\texttt{2}).  It will \alert{increase spacing} between paragraphs and list items.

    \item The idea is to make it easier to stretch slide contents, \emph{without} littering the code with \mintinline{latex}{\vspace{...}} commands everywhere.

    \item This slide uses \texttt{stretch=3.5}.
\end{itemize}

\end{frame}

\begin{frame}[fragile]{Layout: Two-column layout}
To quickly get a two-column layout, you can use:\medskip
    
    \mint{latex}|\twocol{First column here.}{Second column here.}|
    \medskip
    
    By default, this will make both columns equally wide, namely \mintinline{latex}{0.475\linewidth}.
    
    \bigskip
    \twocol[0.3]{An optional argument can be used to specify a different factor for the first column.  Here, I used \texttt{0.3}.}{The second column will automatically expand so that the two columns combined take up \mintinline{latex}{0.95\linewidth}. \textit{(This is less than~1 so that there is some padding between them.)}}
\end{frame}


\section{Colors}

\begin{frame}{Colors}
  This theme defines the following colors that you can use anywhere in your presentation. They are all based on the color values in JU's official Graphic Manual.
  \bigskip
  
  \centering
  \begin{tabular}{ >{\raggedleft\arraybackslash} m{3cm} m{1cm}  >{\raggedleft\arraybackslash} m{3cm} m{1cm} }
    \tt ur2Rouge & \colorbox{ur2Rouge}{\makebox(14,14){~}} &  \tt ur2Jaune & \colorbox{ur2Jaune}{\makebox(14,14){~}} \\
    \tt ur2Gris & \colorbox{ur2Gris}{\makebox(14,14){~}} & \tt ur2Turquoise & \colorbox{ur2Turquoise}{\makebox(14,14){~}} \\
    &&
    \tt ur2Bleu & \colorbox{ur2Bleu}{\makebox(14,14){~}} \\
    &&
    \tt ur2GrisClair & \colorbox{ur2GrisClair}{\makebox(14,14){~}} \\
  \end{tabular}
\end{frame}

\subsection{Tints}

% \begin{frame}{Colors}
%   Different color tints are also defined.
%   \bigskip
  
%   \centering
%  \begin{tabular}{ >{\raggedleft\arraybackslash} m{3.5cm} m{1cm}  >{\raggedleft\arraybackslash} m{3.5cm} m{1cm} }
%     \tt juPurpleDarker & \colorbox{juPurpleDarker}{\makebox(14,14){~}} & & \\
%     \tt juPurpleFade1 & \colorbox{juPurpleFade1}{\makebox(14,14){~}} & \tt juGreyFade1 & \colorbox{juGreyFade1}{\makebox(14,14){~}} \\
%     \tt juPurpleFade2 & \colorbox{juPurpleFade2}{\makebox(14,14){~}} & \tt juGreyFade2 & \colorbox{juGreyFade2}{\makebox(14,14){~}} \\
%     \tt juPurpleFade3 & \colorbox{juPurpleFade3}{\makebox(14,14){~}} & \tt juGreyFade3 & \colorbox{juGreyFade3}{\makebox(14,14){~}} \\
%   \end{tabular}
% \end{frame}

% \begin{frame}{Colors}
%   Color tints for the secondary colors.
%   \bigskip
  
%   \centering
%   \begin{tabular}{ >{\raggedleft\arraybackslash} m{3.5cm} m{1cm}  >{\raggedleft\arraybackslash} m{3.5cm} m{1cm} }
%     \tt juYellowFade1 & \colorbox{juYellowFade1}{\makebox(14,14){~}} & \tt juTurquoiseFade1 & \colorbox{juTurquoiseFade1}{\makebox(14,14){~}} \\
%     \tt juYellowFade2 & \colorbox{juYellowFade2}{\makebox(14,14){~}} & \tt juTurquoiseFade2 & \colorbox{juTurquoiseFade2}{\makebox(14,14){~}} \\
%     \tt juYellowFade3 & \colorbox{juYellowFade3}{\makebox(14,14){~}} & \tt juTurquoiseFade3 & \colorbox{juTurquoiseFade3}{\makebox(14,14){~}} \\[2em]
%     \tt juDarkBlueFade1 & \colorbox{juDarkBlueFade1}{\makebox(14,14){~}} & & \\
%     \tt juDarkBlueFade2 & \colorbox{juDarkBlueFade2}{\makebox(14,14){~}} & & \\
%     \tt juDarkBlueFade3 & \colorbox{juDarkBlueFade3}{\makebox(14,14){~}} & & \\
    
%   \end{tabular}
% \end{frame}

\subsection{Usage}

\begin{frame}[fragile]{Using Colors}

Colors can be used with any commands, for example:
\begin{center}
    \begin{tabular}{ll}
    \toprule
        \mintinline{latex}{\textcolor{ur2Jaune}{Lorem ipsum}} & \textcolor{ur2Jaune}{Lorem ipsum} \\
    \bottomrule
    \end{tabular}
\end{center}

\medskip
Some predefined commands that use these colors:

\begin{center}
    \begin{tabular}{ll}
    \toprule
        \mintinline{latex}{\alert{Lorem ipsum}} & \alert{Lorem ipsum} \\
        \mintinline{latex}{\alertExample{Lorem ipsum}} & \alertExample{Lorem ipsum} \\
        \mintinline{latex}{\highlight{Lorem ipsum}} & \highlight{Lorem ipsum} \\

    \bottomrule
    \end{tabular}
\end{center}

\medskip
The colors are also used in different predefined \hyperlink{environments}{\textcolor{ur2Bleu}{environments}} and \hyperlink{blocks}{\textcolor{ur2Bleu}{blocks}}.

\end{frame}


\section{Fonts}
\label{sec:fonts}

\subsection{Selecting a font theme}

\begin{frame}[fragile,label=selectfonts]{Fonts}

JU's Graphic Manual uses \alert{BentonSans} and \alert{Scala (Sans) OT} as their main fonts, but they are not free and not available on Overleaf.\medskip

  Therefore, this theme provides a number of alternatives:
  
      \begin{itemize}
        \item \alert{\texttt{arial}} uses the recommended web font from JU's Graphic Manual, which is plain old Arial.
        \item \alert{\texttt{franklin}} uses Libre Franklin, which supposedly is the closest free equivalent to BentonSans.  (It doesn't support \textsc{small caps} though.)
        \item \alert{\texttt{source}} uses Adobe's Source fonts, which are slightly more different from BentonSans, but look potentially nicer.
        \item \alert{\texttt{fira}} is the default font of the \textsc{Metropolis} theme, but has no similarity to JU's recommended fonts.
    \end{itemize}
\end{frame}

\begin{frame}[fragile]{Selecting a Font Theme}

Select a font theme with the option \texttt{typeface=<...>}:

\begin{center}
\begin{adjustbox}{max width=\textwidth}
\begin{tabular}{llll}
\toprule
\bf Typeface & \bf Sans font & \bf Mono font & \bf Serif font \\
\midrule
    \texttt{arial} & Arial/Inter & Courier New & Times New Roman \\
%    \multicolumn{4}{l}{~~~\footnotesize\itshape --- Recommended ``web fonts'' from JU's Graphic Manual, but a bit plain.} \\[.4em]
    \texttt{franklin} & Libre Franklin & Hack & Libre Baskerville \\
%    \multicolumn{4}{l}{~~~\footnotesize\itshape --- Closest free equivalent to Benton Sans, but doesn't have \textsc{small caps}.} \\[.4em]
    \texttt{source} & Source Sans Pro & Source Code Pro & Source Serif Pro \\
%    \multicolumn{4}{l}{~~~\footnotesize\itshape --- Adobe open fonts, still close to Benton Sans.} \\[.4em]
    \texttt{fira} & Fira Sans & Fira Code/Mono & Charter \\
%    \multicolumn{4}{l}{~~~\footnotesize\itshape --- Default fonts of the Metropolis theme; no similarity to JU's fonts.} \\[.4em]
\bottomrule
\end{tabular}
\end{adjustbox}
\end{center}

  \ifboolexpr{bool {xetex} or bool {luatex}}{
  }{
  \metroset{block=fill}
    \begin{alertblock}{Warning}
    You are \textbf{not} currently compiling with XeLaTeX or LuaLaTeX!  Font selection will \textbf{not} work in plain (pdf)\LaTeX{}!
    \end{alertblock}
  }

\bigskip
\begin{block}{Fonts currently in use}
\makeatletter{\f@family}, {\tt\f@family}, {\rmfamily\f@family}\makeatother
\end{block}

\end{frame}

\subsection{Samples}

{\metroset{typeface=arial}
\begin{frame}[fragile]{Font Samples: Arial}

Here's a sample of \texttt{typeface=arial}.

\begin{center}
\begin{tabular}{ll}
\toprule
    Sans (default) & The five boxing wizards jump quickly. \\
    \mintinline{latex}{\textit} & \textit{The five boxing wizards jump quickly.} \\
    \mintinline{latex}{\textbf} & \textbf{The five boxing wizards jump quickly.} \\
    \mintinline{latex}{\textsc} & \textsc{The five boxing wizards jump quickly.} \\
    \mintinline{latex}{\extraboldsans} & {\extraboldsans The five boxing wizards jump quickly.} \\
    \mintinline{latex}{\rmfamily} & {\rmfamily The five boxing wizards jump quickly.} \\
      & {\rmfamily The \textbf{five} \textit{boxing} \textbf{\textit{wizards}} \textsc{jump} \textbf{\textsc{quickly}}.} \\
    \mintinline{latex}{\texttt} & \texttt{The five boxing wizards jump quickly.} \\
\bottomrule
\end{tabular}
\end{center}

  Serif font is not used anywhere by default.

  The \mintinline{latex}{\extraboldsans} variant is only used for the~\mintinline{latex}{\juHeading}.
\end{frame}
}

{\metroset{typeface=franklin}
\begin{frame}[fragile]{Font Samples: Libre Franklin}

Here's a sample of \texttt{typeface=franklin}.

\begin{center}
\begin{tabular}{ll}
\toprule
    Sans (default) & The five boxing wizards jump quickly. \\
    \mintinline{latex}{\textit} & \textit{The five boxing wizards jump quickly.} \\
    \mintinline{latex}{\textbf} & \textbf{The five boxing wizards jump quickly.} \\
    \mintinline{latex}{\textsc} & \textsc{The five boxing wizards jump quickly.} \\
    \mintinline{latex}{\extraboldsans} & {\extraboldsans The five boxing wizards jump quickly.} \\
    \mintinline{latex}{\rmfamily} & {\rmfamily The five boxing wizards jump quickly.} \\
      & {\rmfamily The \textbf{five} \textit{boxing} \textbf{\textit{wizards}} \textsc{jump} \textbf{\textsc{quickly}}.} \\
    \mintinline{latex}{\texttt} & \texttt{The five boxing wizards jump quickly.} \\
\bottomrule
\end{tabular}
\end{center}

  Serif font is not used anywhere by default.

  The \mintinline{latex}{\extraboldsans} variant is only used for the~\mintinline{latex}{\juHeading}.
\end{frame}
}

{\metroset{typeface=source}
\begin{frame}[fragile]{Font Samples: Adobe's Source Fonts}

Here's a sample of \texttt{typeface=source}.

\begin{center}
\begin{tabular}{ll}
\toprule
    Sans (default) & The five boxing wizards jump quickly. \\
    \mintinline{latex}{\textit} & \textit{The five boxing wizards jump quickly.} \\
    \mintinline{latex}{\textbf} & \textbf{The five boxing wizards jump quickly.} \\
    \mintinline{latex}{\textsc} & \textsc{The five boxing wizards jump quickly.} \\
    \mintinline{latex}{\extraboldsans} & {\extraboldsans The five boxing wizards jump quickly.} \\
    \mintinline{latex}{\rmfamily} & {\rmfamily The five boxing wizards jump quickly.} \\
      & {\rmfamily The \textbf{five} \textit{boxing} \textbf{\textit{wizards}} \textsc{jump} \textbf{\textsc{quickly}}.} \\
    \mintinline{latex}{\texttt} & \texttt{The five boxing wizards jump quickly.} \\
\bottomrule
\end{tabular}
\end{center}

  Serif font is not used anywhere by default.

  The \mintinline{latex}{\extraboldsans} variant is only used for the~\mintinline{latex}{\juHeading}.
\end{frame}
}


{\metroset{typeface=fira}
\begin{frame}[fragile]{Font Samples: Fira Fonts}

Here's a sample of \texttt{typeface=fira}.

\begin{center}
\begin{tabular}{ll}
\toprule
    Sans (default) & The five boxing wizards jump quickly. \\
    \mintinline{latex}{\textit} & \textit{The five boxing wizards jump quickly.} \\
    \mintinline{latex}{\textbf} & \textbf{The five boxing wizards jump quickly.} \\
    \mintinline{latex}{\textsc} & \textsc{The five boxing wizards jump quickly.} \\
    \mintinline{latex}{\extraboldsans} & {\extraboldsans The five boxing wizards jump quickly.} \\
    \mintinline{latex}{\rmfamily} & {\rmfamily The five boxing wizards jump quickly.} \\
      & {\rmfamily The \textbf{five} \textit{boxing} \textbf{\textit{wizards}} \textsc{jump} \textbf{\textsc{quickly}}.} \\
    \mintinline{latex}{\texttt} & \texttt{The five boxing wizards jump quickly.} \\
\bottomrule
\end{tabular}
\end{center}

  Serif font is not used anywhere by default.

  The \mintinline{latex}{\extraboldsans} variant is only used for the~\mintinline{latex}{\juHeading}.
\end{frame}
}


\section{Environments}

\subsection{Enumerations}

{\metroset{itemize=colored}
\setbeamertemplate{enumerate items}[mycircle]
\begin{frame}[fragile,stretch,label=environments]{Enumerate environments}
    Combine \mintinline{latex}{\metroset{itemize=colored}} with
    {\mintinline{latex}{\setbeamertemplate{enumerate items}[mycircle]}} to get:
\medskip

    \begin{enumerate}
        \item One
        \begin{enumerate}[a]
            \item alpha
            \begin{enumerate}[i]
                \item foo
                \item bar
                \item baz
            \end{enumerate}
            \item omega
        \end{enumerate}
        \item Two
        \item Three
    \end{enumerate}
\end{frame}
}

\subsection{Quotations}

\begin{frame}[fragile]{Quotations}

    \begin{quote}%
    Beautiful is better than ugly.
    Explicit is better than implicit.
    Simple is better than complex. [\ldots]
    Readability counts.
    \end{quote}
    \attribution{Tim Peters, from \textsc{The Zen of Python}}

\medskip
The quotation above can be produced via:

    {\footnotesize
    \begin{minted}{latex}
    \begin{quote}
      Beautiful is better than ugly.
      Explicit is better than implicit.
      Simple is better than complex. [\ldots]
      Readability counts.
    \end{quote}
    \attribution{Tim Peters, from \textsc{The Zen of Python}}
    \end{minted}
    }

\end{frame}

%\begin{frame}[standout]
%    This is a standout slide.
%    \bigskip\bigskip
%    
%    \normalsize
%    Standout slides are a Metropolis feature activated through the \texttt{[standout]} frame option.
%    \bigskip
%    
%    In contrast to Metropolis defaults, here it will have the logotype headline, footer, and frame numbering.
%\end{frame}


\subsection{Blocks}

{\metroset{block=transparent}
\begin{frame}[label=blocks]{Blocks}
    These are beamer blocks with \texttt{block=transparent}.\medskip

    \begin{block}{This is a regular \texttt{block}.}
    Here is some content.
    \end{block}

    \begin{alertblock}{This is an \texttt{alertblock}.}
    Here is some content.
    \end{alertblock}

    \begin{exampleblock}{This is an \texttt{exampleblock}.}
    Here is some content.
    \end{exampleblock}

    \begin{warningblock}{This is a \texttt{warningblock}.}
    Here is some content.
    \end{warningblock}
\end{frame}
}

{\metroset{block=fill}
\begin{frame}{Blocks}
    These are beamer blocks with \texttt{block=fill}.\medskip

    \begin{block}{This is a regular \texttt{block}.}
    Here is some content.
    \end{block}

    \begin{alertblock}{This is an \texttt{alertblock}.}
    Here is some content.
    \end{alertblock}

    \begin{exampleblock}{This is an \texttt{exampleblock}.}
    Here is some content.
    \end{exampleblock}

    \begin{warningblock}{This is a \texttt{warningblock}.}
    Here is some content.
    \end{warningblock}
\end{frame}
}

\subsection{Code}
\begin{frame}[fragile,stretch]{Code Blocks}
    
    Since I frequently need to show code examples, I defined some styles and custom commands that can be included with:
    
    \mint{latex}|\usepackage{myminted}|
    
    This uses the \href{http://tug.ctan.org/macros/latex/contrib/minted/minted.pdf}{{\small\faExternalLink*}~\texttt{minted}} package to typeset code with automatic syntax highlighting.
    \medskip

    \begin{warningblock}{\faExclamationTriangle{}\enspace{}Important}
    Every frame that contains code must have the \texttt{[fragile]} option set, or compilation will break with many cryptic errors!
    \end{warningblock}
\end{frame}

\begin{frame}[fragile]{Code Blocks}
    I also defined some commands to easily produce code blocks like the following; see the \LaTeX{} source for details:\medskip
    
    \begin{beamercodeblock}\vspace{-.6em}
      \begin{minted}[linenos,fontsize=\footnotesize]{java}
      // Your First Program

      class HelloWorld {
        public static void main(String[] args) {
          System.out.println("Hello, World!"); 
        }
      }
      \end{minted}
    \end{beamercodeblock}
    \beamercaptionblock{\textbf{Listing 1:} A hello world program in Java.}

\end{frame}


\subsection{Bibliography}
\begin{frame}[fragile,stretch]{Bibliography}
    
    I use \texttt{biblatex} with some custom definitions, which I also bundled in their own package:
    
    \mint{latex}|\usepackage{mybiblatex}|
    
    Mainly, this will suppress output of URLs and DOIs, and instead turn the paper title into hyperlinks.  Here's an example:

    \begin{itemize}
        \item \fullcite{iki-aizawa-2020-language}
    \end{itemize}
\end{frame}

\appendix
\metroset{sectionpage=none}
\section{Bibliography}

\begin{frame}{Bibliography}

    \nocite{*}
    \printbibliography[heading=none]
    
    This slide also demonstrates that appendices work \& play nicely with the other features!
\end{frame}

\begin{frame}[plain,standout,noframenumbering]
\urLogoW[height=2.5cm]
\end{frame}

\end{document}
