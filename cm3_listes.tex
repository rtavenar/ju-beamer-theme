\documentclass[10pt]{beamer}

\usetheme[%
    progressbar=frametitle,
    block=fill,
    numbering=fraction,
    footer=crumbs,
    sectionpage=numbered,
    subsectionpage=none,
    titleformattitle=smallcaps,
    titleformatsubtitle=smallcaps,
    %%% new options for UR2 theme:
    maincolor=red,
]{metropolis-ur2}

\usepackage{hyperref}
\hypersetup{
    colorlinks=true,
    linkcolor=.,
    filecolor=ur2Turquoise,
    urlcolor=ur2Bleu,
    pdftitle={M1 MAS - Python - Structures de données},
    pdfpagemode=FullScreen,
}

% \usepackage{biblatex}
% \addbibresource{example.bib}

\usepackage{appendixnumberbeamer}
\usepackage{adjustbox}
\usepackage{booktabs}
\usepackage{fontawesome5}
\usepackage{graphicx}
\usepackage{array}
\usepackage{tabularx}
\usepackage{myminted}
% \usepackage[cache=false]{minted}

\title[M1 MAS -- Python -- Listes]{CM 3 - Listes}
\subtitle{Programmation Python -- Master 1 MAS}
\author{Romain Tavenard}
\date{2022}
\institute{%
\hypersetup{urlcolor=.}
\makebox[2.2ex][c]{\faEnvelope}\enspace\href{mailto:romain.tavenard@univ-rennes2.fr}{\texttt{romain.tavenard@univ-rennes2.fr}}\\%
% \makebox[2.2ex][c]{\faHome}\enspace\url{https://rtavenar.github.io/}%
}

\begin{document}

\maketitle

% \section{Les listes}

\begin{frame}[fragile]{Parcours de listes (1/2)}  
  3 parcours de listes en Python :

  \begin{beamercodeblock}
    \begin{minted}[fontsize=\footnotesize]{python}
      ma_liste = [4, 3, 3]

      for valeur in ma_liste:  # Parcours par valeur
        print(valeur)
      
      for i in range(len(ma_liste)):  # Parcours par indice
        print(i)
      
      for i, valeur in enumerate(ma_liste):  # Parcours par indice ET valeur
        print(i, valeur)
    \end{minted}
  \end{beamercodeblock}
\end{frame}

\begin{frame}[fragile]{Parcours de listes (2/2)}  
  On dispose d'une liste \mintinline{python}|li = [5, 3, 7, 2]|.
  Quel parcours utiliser pour\dots

  \begin{itemize}
    \item Déterminer le minimum de la liste \mintinline{python}|li| ? \pause \alert{Par valeur}
    \item Déterminer la position du minimum de la liste \mintinline{python}|li| ? \pause \alert{Par indice ET valeur}
    \item Créer une nouvelle liste définie comme \mintinline{python}|[f(5), f(3), f(7), f(2)]| ? \pause \alert{Par valeur}
    \item Créer une nouvelle liste définie comme \mintinline{python}|[f(5, 3), f(3, 7), f(7, 2)]| ? \pause \alert{Par indice}
  \end{itemize}
\end{frame}

\begin{frame}[fragile]{(arg)max/(arg)min d'une liste}
  \begin{beamercodeblock}
    \begin{minted}[fontsize=\footnotesize]{python}
      max_val = PAS_BEAUCOUP_DU_TOUT  # Mais quoi ?
      for val in ma_liste:
        if val > max_val:
          max_val = val
      # À la fin de la boucle, la valeur maximale est stockée dans `max_val`
    \end{minted}
  \end{beamercodeblock}

  \pause
  
  \begin{beamercodeblock}
    \begin{minted}[fontsize=\footnotesize]{python}
      max_val = PAS_BEAUCOUP_DU_TOUT  # Mais quoi ?
      indice_max = None
      for i, val in enumerate(ma_liste):
        if val > max_val:
          max_val = val
          indice_max = i
      # À la fin de la boucle, l'argmax est stocké dans `indice_max`
    \end{minted}
  \end{beamercodeblock}
\end{frame}

\begin{frame}[fragile]{Exercice}
  \begin{quote}
    Écrivez une fonction qui prend en entrée une liste de chaînes de caractères et retourne la position de la chaîne de la liste qui contient le plus grand nombre de `"A"' majuscules.
  \end{quote}
\end{frame}

\begin{frame}[fragile]{Parcours de listes : piège}  
  On ne doit \alert{jamais} parcourir une liste pendant qu'on lui insère / supprime des éléments.

  Alternatives :

  \begin{beamercodeblock}
    \begin{minted}[fontsize=\footnotesize]{python}
      nouvelle_liste = []
      for v in ma_liste:
        nouvelle_liste.append(v)
        if une_certaine_condition:
          nouvelle_liste.append(valeur_a_inserer)
    \end{minted}
  \end{beamercodeblock}
  
  \begin{beamercodeblock}
    \begin{minted}[fontsize=\footnotesize]{python}
      nouvelle_liste = []
      for v in ma_liste:
        if une_certaine_condition:
          nouvelle_liste.append(v)
        # Sinon, on "supprime" la valeur `v`
    \end{minted}
  \end{beamercodeblock}
\end{frame}

\begin{frame}[fragile]{Exercice}
  \begin{quote}
    Écrivez une fonction qui prend en entrée une liste d'entiers et retourne une liste où chaque entier \mintinline{python}|n| a été répété exactement \mintinline{python}|n| fois.

    Par exemple, si on fournit en entrée la liste \mintinline{python}|[2, 0, 3, 1]|, la fonction devra retourner la liste \mintinline{python}|[2, 2, 3, 3, 3, 1]|.
  \end{quote}
\end{frame}

\begin{frame}[fragile]{Recopie de liste}  
  Que vaut \mintinline{python}|li| à la fin du code suivant ?

  \begin{beamercodeblock}
    \begin{minted}[fontsize=\footnotesize]{python}
      li = [3, 4, 1]
      nouvelle_liste = li
      nouvelle_liste.append(4)
    \end{minted}
  \end{beamercodeblock}

  \pause

  Pour éviter les mésaventures :
  
  \begin{beamercodeblock}
    \begin{minted}[fontsize=\footnotesize]{python}
      li = [3, 4, 1]
      nouvelle_liste = list(li)
      nouvelle_liste.append(4)
    \end{minted}
  \end{beamercodeblock}

  \pause

  \mintinline{python}|list(li)| permet de faire une \alert{copie superficielle} de \mintinline{python}|li|.

  \emph{Les copies profondes sont en dehors du scope de ce cours. Référez-vous au module \mintinline{python}|deepcopy| si besoin.}
\end{frame}

\begin{frame}[fragile]{Structures emboîtées}  
  Comment représenter une matrice avec une / des liste(s) ?

  $$
  \left(
    \begin{matrix}
      1 & 2 & 3 \\
      4 & 5 & 6
    \end{matrix}
  \right)
  $$

  \pause

  \begin{beamercodeblock}
    \begin{minted}[fontsize=\footnotesize]{python}
      ma_matrice = [[1, 2, 3],
                    [4, 5, 6]]
    \end{minted}
  \end{beamercodeblock}

  Et, alors, comment accéder à l'élément d'indice $(i, j)$ ? \pause \mintinline{python}|ma_matrice[i][j]|.
\end{frame}

\begin{frame}{Exercice}
  \begin{quote}
    Écrivez une fonction qui prend en entrée un entier $n$ et retourne la matrice identité de dimension $n \times n$ représentée à l'aide de listes.
  \end{quote}
\end{frame}


\end{document}
