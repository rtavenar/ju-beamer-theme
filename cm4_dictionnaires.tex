\documentclass[10pt]{beamer}

\usetheme[%
    progressbar=frametitle,
    block=fill,
    numbering=fraction,
    footer=crumbs,
    sectionpage=numbered,
    subsectionpage=none,
    titleformattitle=smallcaps,
    titleformatsubtitle=smallcaps,
    %%% new options for UR2 theme:
    maincolor=red,
]{metropolis-ur2}

\usepackage{hyperref}
\hypersetup{
    colorlinks=true,
    linkcolor=.,
    filecolor=ur2Turquoise,
    urlcolor=ur2Bleu,
    pdftitle={M1 MAS - Python - Structures de données},
    pdfpagemode=FullScreen,
}

% \usepackage{biblatex}
% \addbibresource{example.bib}

\usepackage{appendixnumberbeamer}
\usepackage{adjustbox}
\usepackage{booktabs}
\usepackage{fontawesome5}
\usepackage{graphicx}
\usepackage{array}
\usepackage{tabularx}
\usepackage{myminted}
% \usepackage[cache=false]{minted}

\title[M1 MAS -- Python -- Dictionnaires]{CM 4 - Dictionnaires et Sets}
\subtitle{Programmation Python -- Master 1 MAS}
\author{Romain Tavenard}
\date{2022}
\institute{%
\hypersetup{urlcolor=.}
\makebox[2.2ex][c]{\faEnvelope}\enspace\href{mailto:romain.tavenard@univ-rennes2.fr}{\texttt{romain.tavenard@univ-rennes2.fr}}\\%
% \makebox[2.2ex][c]{\faHome}\enspace\url{https://rtavenar.github.io/}%
}

\begin{document}

\maketitle

\section{Les dictionnaires}

\begin{frame}[fragile]{Parcours de dictionnaires (1/2)}  
  3 parcours de dictionnaires en Python :

  \begin{beamercodeblock}
    \begin{minted}[fontsize=\footnotesize]{python}
      mon_dico = {"a": 4, "c": 3, "z": 3}

      for val in mon_dico.values():  # Parcours par valeur
        print(val)

      for cle in mon_dico.keys():  # Parcours par clé
        print(cle)
      
      for cle, val in mon_dico.items():  # Parcours par clé ET valeur
        print(cle, valeur)
    \end{minted}
  \end{beamercodeblock}

  \pause

  Question subsidiaire : que vaut \mintinline{python}|x| dans le parcours suivant ?

  \begin{beamercodeblock}
    \begin{minted}[fontsize=\footnotesize]{python}
      for x in mon_dico:
        print(x)
    \end{minted}
  \end{beamercodeblock}
\end{frame}

\begin{frame}[fragile]{Parcours de dictionnaires (2/2)}  
  On dispose d'un dictionnaire \mintinline{python}|d = {"a": 4, "c": 3, "z": 3}|.

  Quel parcours utiliser pour\dots

  \begin{itemize}
    \item Déterminer le minimum des valeurs du dictionnaire \mintinline{python}|d| ? \pause \alert{Par valeur}
    \item Déterminer la clé correspondant du minimum des valeurs du dictionnaire \mintinline{python}|d| ? \pause \alert{Par clé ET valeur}
    \item Créer un nouveau dictionnaire défini comme \mintinline{python}|{"a": f(4), "c": f(3), "z": f(3)}| ? \pause \alert{Par clé ET valeur}
  \end{itemize}
\end{frame}

\begin{frame}[fragile]{(arg)max/(arg)min d'un dictionnaire}
  \begin{beamercodeblock}
    \begin{minted}[fontsize=\footnotesize]{python}
      max_val = PAS_BEAUCOUP_DU_TOUT  # Mais quoi ?
      for val in mon_dico.values():
        if val > max_val:
          max_val = val
      # À la fin de la boucle, la valeur maximale est stockée dans `max_val`
    \end{minted}
  \end{beamercodeblock}

  \pause
  
  \begin{beamercodeblock}
    \begin{minted}[fontsize=\footnotesize]{python}
      max_val = PAS_BEAUCOUP_DU_TOUT  # Mais quoi ?
      cle_max = None
      for cle, val in mon_dico.items():
        if val > max_val:
          max_val = val
          cle_max = cle
      # À la fin de la boucle, l'argmax est stocké dans `cle_max`
    \end{minted}
  \end{beamercodeblock}
\end{frame}

\begin{frame}[fragile]{Les clés de dictionnaires}
  Un dictionnaire peut avoir pour clés :
  \begin{itemize}
    \item des chaînes de caractères
    \item des entiers
    \item des flottants (mais c'est une mauvaise idée)
    \item des tuples (ex : \mintinline{python}|(1, 3)|)
    \item \alert{mais PAS des listes ni des dictionnaires}
  \end{itemize}
\end{frame}

\begin{frame}[fragile]{Accéder à une valeur via sa clé}
  Si l'on souhaite accéder à la valeur associée à la clé \mintinline{python}|"xyz"|, pas besoin de parcourir !

  \pause
  
  \begin{beamercodeblock}
    \begin{minted}[fontsize=\footnotesize]{python}
      # Option 1
      if "xyz" in mon_dico.keys():
        print(mon_dico["xyz"])
      else:
        print("Clé 'xyz' absente du dico")

      # Option 2 (A TOUJOURS FAVORISER)
      print(mon_dico.get("xyz", "Clé 'xyz' absente du dico"))
    \end{minted}
  \end{beamercodeblock}

\end{frame}

\begin{frame}{Exercice}
  \begin{quote}
    Écrire une fonction qui prend en entrée une chaîne de caractère et retourne le caractère le plus fréquent de cette chaîne.
  \end{quote}
\end{frame}

\begin{frame}[fragile]{Pièges}
  Comme pour les listes, on devra prendre garde aux pièges suivants :
  
  \begin{itemize}
    \item On ne doit \alert{jamais} parcourir un dictionnaire pendant qu'on lui insère / supprime des paires clés-valeurs ;
    \item Si l'on souhaite effectuer une copie de dictionnaire, il est bon d'utiliser la fonction \mintinline{python}|dict()| (se référer au module \mintinline{python}|deepcopy| si besoin de copies profondes).
  \end{itemize}
\end{frame}

\begin{frame}{Exercice}

  \begin{quote}
    On a vu dans le cours précédent qu'une matrice peut être représentée comme une liste de listes.
    Proposez une structure de données plus adaptée pour représenter les matrices creuses 
    (c'est-à-dire les matrices dans lesquelles de nombreuses valeurs valent 0), comme dans l'exemple suivant :

        $$
        \left(
          \begin{matrix}
            0 & 0 & 0 \\
            0 & 1 & 0
          \end{matrix}
        \right)
        $$

    Comment faut-il faire pour accéder à l'élément d'indice $(i, j)$ d'une matrice ainsi représentée ?
  \end{quote}
\end{frame}

\section{Les sets}

\begin{frame}[fragile]{Les sets : Vision simplifiée}
  Un \emph{set} peut être vu comme un \alert{dictionnaire sans valeurs} (que des clés).

  Quelle utilité ?
  \begin{itemize}
    \item stocker des ensembles sans répétition
    \item construction simple à partir de liste ou chaîne :
          \begin{beamercodeblock}
            \begin{minted}[fontsize=\footnotesize]{python}
              set([1, 3, 2, 1])  # {1, 3, 2}
              set("abcdcb")      # {"a", "b", "c", "d"}
            \end{minted}
          \end{beamercodeblock}
    \item méthodes sur les \emph{sets} :
    \begin{itemize}
      \item insertion    \mintinline{python}|mon_set.add(item)| \\ (si doublon, pas d'insertion)
      \item union        \mintinline{python}+{"a", "b"} | {"b", "c"}+ ou \mintinline{python}|{"a", "b"}.union({"b", "c"})|
      \item intersection \mintinline{python}|{"a", "b"} & {"b", "c"}| ou \mintinline{python}|{"a", "b"}.intersection({"b", "c"})|
    \end{itemize}
  \end{itemize}
\end{frame}

\begin{frame}[fragile]{Exercice}
  \begin{quote}
    Écrire une fonction qui prend en entrée une chaîne de caractère et retourne le caractère le plus fréquent de cette chaîne (sans utiliser de dictionnaire).
  \end{quote}
\end{frame}

\section{Structures imbriquées}


\begin{frame}[fragile]{Structures imbriquées : symptômes}
  En pratique, les vraies données sont souvent du type :

  \begin{beamercodeblock}
    \begin{minted}[fontsize=\footnotesize]{python}
      [
        {
          "address": {
            "building": "1449",
            "street": "Nostrand Avenue",
            "zipcode": "11226"
          },
          "grades": [
            {"date": "2014-10-21T02:00:00", "grade": "B"},
            # [...]
            {"date": "2011-12-14T01:00:00", "grade": "A"}
          ],
          "name": "Nostrand Donut Shop"
        },
        # [...]
      ]
    \end{minted}
  \end{beamercodeblock}
\end{frame}


\begin{frame}[fragile]{Structures imbriquées : remèdes}
  Pour visualiser :

  \begin{beamercodeblock}
    \begin{minted}[fontsize=\footnotesize]{python}
      from pprint import pprint

      pprint(ma_structure_compliquee)
    \end{minted}
  \end{beamercodeblock}
 
  Pour accéder aux éléments :

  \begin{beamercodeblock}
    \begin{minted}[fontsize=\footnotesize]{python}
      pprint(ma_structure_compliquee[0])  # Visualiser un niveau, 
                                          # comprendre son type
    \end{minted}
  \end{beamercodeblock}

  \pause

  \begin{beamercodeblock}
    \begin{minted}[fontsize=\footnotesize]{python}
      pprint(ma_structure_compliquee[0]["address"])  # Visualiser un niveau, 
                                                     # comprendre son type
    \end{minted}
  \end{beamercodeblock}
 \end{frame}

 \begin{frame}[fragile]{Exercice}
  \begin{quote}
    Écrivez un morceau de code qui permet de stocker l'ensemble des notes (\mintinline{python}|"A"|, \mintinline{python}|"B"|, \emph{etc.}) attribuées (sans doublon) dans le jeu de données ci-dessous.
  \end{quote}

  \begin{beamercodeblock}
    \begin{minted}[fontsize=\footnotesize]{python}
      [
        {
          "address": {"building": "1449", "street": "Nostrand Avenue"},
          "grades": [
            {"date": "2014-10-21T02:00:00", "grade": "B"},
            # [...]
            {"date": "2011-12-14T01:00:00", "grade": "A"}
          ],
          "name": "Nostrand Donut Shop"
        },
        # [...]
      ]
    \end{minted}
  \end{beamercodeblock}
\end{frame}

\end{document}
