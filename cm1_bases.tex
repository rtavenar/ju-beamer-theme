\documentclass[10pt]{beamer}

\usetheme[%
    progressbar=frametitle,
    block=fill,
    numbering=fraction,
    footer=crumbs,
    sectionpage=numbered,
    subsectionpage=none,
    titleformattitle=smallcaps,
    titleformatsubtitle=smallcaps,
    %%% new options for UR2 theme:
    maincolor=red,
]{metropolis-ur2}

\usepackage{hyperref}
\hypersetup{
    colorlinks=true,
    linkcolor=.,
    filecolor=ur2Turquoise,
    urlcolor=ur2Bleu,
    pdftitle={M1 MAS - Python - Bases},
    pdfpagemode=FullScreen,
}

% \usepackage{biblatex}
% \addbibresource{example.bib}

\usepackage{appendixnumberbeamer}
\usepackage{adjustbox}
\usepackage{booktabs}
\usepackage{fontawesome5}
\usepackage{graphicx}
\usepackage{array}
\usepackage{tabularx}
\usepackage{myminted}
% \usepackage[cache=false]{minted}

\title[M1 MAS -- Python -- Bases]{CM 1 - Bases}
\subtitle{Programmation Python -- Master 1 MAS}
\author{Romain Tavenard}
\date{2022}
\institute{%
\hypersetup{urlcolor=.}
\makebox[2.2ex][c]{\faEnvelope}\enspace\href{mailto:romain.tavenard@univ-rennes2.fr}{\texttt{romain.tavenard@univ-rennes2.fr}}\\%
% \makebox[2.2ex][c]{\faHome}\enspace\url{https://rtavenar.github.io/}%
}

\begin{document}

\maketitle

\begin{frame}[fragile]{Dans ce cours}  
  \begin{itemize}
    \item Environnement de développement : VS Code
    \item 1 TD (/ projet) = 1 Script Python
    \item Structure de votre code
    \begin{beamercodeblock}
      \begin{minted}[fontsize=\footnotesize]{python}
      # 1. Les imports
  
      # 2. Les fonctions
  
      # 3. Les tests
      \end{minted}
    \end{beamercodeblock}
    \item Polycopié associé : \url{https://rtavenar.github.io/poly_python/}
  \end{itemize}

\end{frame}

\section{Variables \& Structures de contrôle}

\begin{frame}[fragile]{Variables}  
  Pour définir une variable en Python :
  \begin{beamercodeblock}
    \begin{minted}[fontsize=\footnotesize]{python}
    chaine_exemple = "une valeur"
    entier_exemple = 123
    autre_entier = entier_exemple ** 2  # Élévation à la puissance
    encore_un_autre = entier_exemple % 2  # Opérateur "modulo"
    un_flottant_ici = entier_exemple / 2  # Division à virgule
    un_entier_la = entier_exemple // 2  # Division entière
    \end{minted}
  \end{beamercodeblock}
  Le type dépend de la valeur affectée.

  Il vous est demandé d'utiliser des \alert{noms de variable explicites}.
\end{frame}

\begin{frame}[fragile]{Structures conditionnelles}  
  Syntaxe du \mintinline{python}|if ... elif ... else|
  \begin{beamercodeblock}
    \begin{minted}[fontsize=\footnotesize]{python}
    if condition:
      # Code si Vrai
    elif autre_condition:
      # Code si Vrai
    else:
      # Code sinon
    \end{minted}
  \end{beamercodeblock}
  Attention à \alert{l'indentation} !

  Les clauses \mintinline{python}|elif| et \mintinline{python}|else| sont optionnelles.

  \mintinline{python}|condition| et \mintinline{python}|autre_condition| sont des expressions booléennes.
\end{frame}

\begin{frame}[fragile]{Expressions booléennes}
  Exemples d'expressions booléennes :
  \begin{itemize}
    \item \mintinline{python}|x > 10|
    \item \mintinline{python}|y <= 8|
    \item \mintinline{python}|z == "a"|
    \item \mintinline{python}|a != "a"|
    \item \mintinline{python}|(a > 10) or (b == 5)|
    \item \mintinline{python}|(a > 10) or c| \\ où \mintinline{python}|c| est une variable de type \mintinline{python}|bool|
  \end{itemize}
\end{frame}

\begin{frame}{Exercice}
  \begin{quote}
    Écrivez une expression conditionnelle, qui à partir d'une température d'eau stockée dans une variable \mintinline{python}|c| affiche si l'eau à cette température est à l'état liquide, solide ou gazeux.
  \end{quote}
\end{frame}

\begin{frame}[fragile]{Boucles for}  
  Syntaxe du \mintinline{python}|for|
  \begin{beamercodeblock}
    \begin{minted}[fontsize=\footnotesize]{python}
      for i in range(10):  # i allant de 0 à 9
        # Code à répéter pour toutes les valeurs de i
    
      for i in range(5, 20):  # i allant de 5 à 19
        # Code à répéter pour toutes les valeurs de i
    \end{minted}
  \end{beamercodeblock}
  Attention à \alert{l'indentation} !

  En Python, la boucle \mintinline{python}|for| permet d'itérer sur \alert{une séquence de valeurs}.
\end{frame}

\begin{frame}[fragile]{Boucles while}  
  Syntaxe du \mintinline{python}|while|
  \begin{beamercodeblock}
    \begin{minted}[fontsize=\footnotesize]{python}
      while condition:
        # Code à répéter tant que condition vaut Vrai
    \end{minted}
  \end{beamercodeblock}
  Attention à \alert{l'indentation} !

  \mintinline{python}|condition| est une expression booléenne (comme pour le \mintinline{python}|if|).
\end{frame}

\begin{frame}[fragile]{Quelle boucle choisir ?}  
  Pour choisir entre  \mintinline{python}|for| et \mintinline{python}|while|
  
  \begin{itemize}
    \item Si nombre d'itérations fixé $\Rightarrow$ \mintinline{python}|for|
    \item Si on veut répéter le même traitement pour toutes les valeurs d'une séquence (\mintinline{python}|range|, liste, dictionnaire) $\Rightarrow$ \mintinline{python}|for|
    \item Sinon $\Rightarrow$ \mintinline{python}|while|
  \end{itemize}
\end{frame}

\begin{frame}{Exercice}
  \begin{quote}
    Écrivez une fonction en Python qui affiche tous les termes plus petits que 1000 de la suite $(u_n)$ définie comme :

$$
\begin{array}{rcc}u_0 & = & 2 \\
\forall n \geq 1, \, u_n & = & u_{n-1}^2\end{array}
$$
  \end{quote}
\end{frame}

\section{Fonctions}

\begin{frame}[fragile]{Définir une fonction en Python (1/2)}  
  Une fonction a (peut avoir) des paramètres et une valeur de retour :
    
    \begin{beamercodeblock}
      \begin{minted}[fontsize=\footnotesize]{python}
      def affiche_pair(n):
        for i in range(n):
          if i % 2 == 0:
            print(i)
      \end{minted}
    \end{beamercodeblock}

  \begin{itemize}
    \item Nom de la fonction : \mintinline{python}|affiche_pair|
    \item Paramètre(s) / Argument(s) : \mintinline{python}|n| (1 seul paramètre / argument)
    \item Valeur de retour : Pas de valeur de retour ici (pas de \mintinline{python}|return|)
  \end{itemize}
\end{frame} 

\begin{frame}[fragile]{Définir une fonction en Python (2/2)}  
  Une fonction a (peut avoir) des paramètres et une valeur de retour :
    
    \begin{beamercodeblock}
      \begin{minted}[fontsize=\footnotesize]{python}
      def est_pair(i):
        return (i % 2 == 0)
      \end{minted}
    \end{beamercodeblock}

  \begin{itemize}
    \item Nom de la fonction : \mintinline{python}|est_pair|
    \item Paramètre(s) / Argument(s) : \mintinline{python}|i| (1 seul paramètre / argument)
    \item Valeur de retour : valeur booléenne indiquant si \mintinline{python}|i| est pair
  \end{itemize}
\end{frame} 

\begin{frame}[fragile]{Appel de fonction existante}  
  Deux façons d'appeler une fonction en Python :
  \begin{enumerate}
    \item Si elle retourne une valeur que l'on souhaite utiliser :
    
    \begin{beamercodeblock}
      \begin{minted}[fontsize=\footnotesize]{python}
      x = est_pair(5)  # x est de type bool

      # ou bien
      print(est_pair(5))

      # ou encore
      if est_pair(5):
        print("Youhou")
      \end{minted}
    \end{beamercodeblock}

    \item Si elle ne retourne rien ou que sa valeur de \alert{retour} ne nous intéresse pas :
    
    \begin{beamercodeblock}
      \begin{minted}[fontsize=\footnotesize]{python}
      affiche_pair(5)
      \end{minted}
    \end{beamercodeblock}
  \end{enumerate}
\end{frame}

\begin{frame}[fragile]{Les paramètres optionnels}  
  Une fonction peut avoir des paramètres optionnels :
    
    \begin{beamercodeblock}
      \begin{minted}[fontsize=\footnotesize]{python}
      def est_multiple(i, p=2):
        return (i % p == 0)
      
      print(est_multiple(9, 3))  # Affiche True
      print(est_multiple(9))     # Affiche False
      \end{minted}
    \end{beamercodeblock}

  Les paramètres optionnels sont toujours placés \alert{après} les paramètres non optionnels.
\end{frame}

\begin{frame}{Exercice}
  \begin{quote}
    Écrivez une fonction en Python qui prenne en argument une longueur \mintinline{python}|long| et retourne l'aire du triangle équilatéral de côté \mintinline{python}|long|.
  \end{quote}
\end{frame}

\section{Chaînes de caractères}

\begin{frame}[fragile]{Le type str}  
  Il existe en Python de nombreuses fonctions pour manipuler simplement les chaînes de caractères :
    
    \begin{beamercodeblock}
      \begin{minted}[fontsize=\footnotesize]{python}
      x = "une chaîne de caractères "
      y = "et une autre"
      print(len(x))  # Affiche 25
      print(x + y)   # Affiche une chaîne de caractères et une autre
      print(y * 3)   # Affiche et une autreet une autreet une autre

      age = 25
      print(f"L'âge est de {age} ans.")  # Affiche L'âge est de 25 ans.
      \end{minted}
    \end{beamercodeblock}
\end{frame} 

\begin{frame}[fragile]{Les méthodes de la classe str}
  \begin{tabular}{ll}
    \hline
    Code & Action \\
    \hline
    \mintinline{python}|ch.count(sub)| & Compte les occurrences \\
    \mintinline{python}|ch.endswith(suffix)| & Teste la fin de la chaîne \\
    \mintinline{python}|ch.startswith(prefix)| & Teste le début de la chaîne \\
    \mintinline{python}|ch.find(sub)| & Indice du début de la première occurrence \\
    \mintinline{python}|ch.rfind(sub)| & Indice du début de la dernière occurrence \\
    \mintinline{python}|ch.lower()| & \mintinline{python}|ch| en minuscules \\
    \mintinline{python}|ch.upper()| & \mintinline{python}|ch| en majuscules \\
    \mintinline{python}|ch.replace(old, new)| & \begin{tabular}{l}Copie de \mintinline{python}|ch| dans laquelle toutes les\\ occurrences de \mintinline{python}|old| ont été remplacées \\par \mintinline{python}|new|\end{tabular} \\
    \mintinline{python}|ch.split(sep=None)| & Découpe \mintinline{python}|ch| à chaque occurrence de \mintinline{python}|sep| \\
    \hline
  \end{tabular}
\end{frame} 

\begin{frame}{Exercice}
  \begin{quote}
    Écrivez une fonction qui prenne en argument deux chaînes de caractères \mintinline{python}|s| et \mintinline{python}|prefix| et retourne le nombre de mots de la chaîne \mintinline{python}|s| qui débutent par la chaîne \mintinline{python}|prefix|.
  \end{quote}
\end{frame}


\end{document}
